\begin{nvimstyle}
#ifndef ELGAMAL_HPP
#define ELGAMAL_HPP

#include "protocol.hpp"

#include <boost/multiprecision/cpp_int.hpp>
#include <boost/random/mersenne_twister.hpp>

using BigInt = boost::multiprecision::cpp_int;

namespace CRYPTO
{
class ElGamal final : public Protocol
{
public:
	explicit ElGamal(unsigned int key_bits = 2048);

	void init() override;

	QString encrypt(const QString& plaintext) override;
	QString decrypt(const QString& ciphertext) override;

private:
	void generateParameters();
	void generateKeys();
	BigInt generatePrime(unsigned int bits, boost::random::mt19937& rng);

	unsigned int m_key_bits; // Длина ключа в битах

	// Публичные параметры, общие для всех
	BigInt m_p; // Большое простое число (модуль)
	BigInt m_g; // Порождающий элемент (генератор)

	// Ключи
	BigInt m_d; // Закрытый ключ (секретное число, 1 < d < p-1)
	BigInt m_e; // Открытый ключ (e = g^d mod p)
};

} // namespace CRYPTO

#endif // ELGAMAL_HPP
\end{nvimstyle}
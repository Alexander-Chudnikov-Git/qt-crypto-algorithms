\begin{nvimstyle}
QString RSA::encrypt(const QString& plaintext)
{
    // 1. Преобразуем открытый текст в байты
    QByteArray bytes = plaintext.toUtf8();
    
    // 2. Преобразуем байты в шестнадцатеричную строку.
    QString hex_plaintext = bytes.toHex();
    
    // 3. Преобразуем шестнадцатеричную строку в BigInt.
    BigInt message("0x" + hex_plaintext.toStdString());

    // 4. Проверяем, что сообщение меньше модуля n
    if (message >= m_m)
    {
        throw std::runtime_error("Message is too large for the current key size.");
    }

    // 5. Шифруем: c = m^e mod n
    BigInt ciphertext = boost::multiprecision::powm(message, m_e, m_m);

    // 6. Преобразуем зашифрованное число в строку в шестнадцатеричном формате
    return QString::fromStdString(ciphertext.str(0, std::ios_base::hex));
}
\end{nvimstyle}
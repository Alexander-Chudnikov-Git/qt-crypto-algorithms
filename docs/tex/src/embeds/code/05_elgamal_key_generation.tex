\begin{nvimstyle}
void ElGamal::generateParameters()
{
	boost::random::mt19937 rng(std::chrono::high_resolution_clock::now().time_since_epoch().count());

	// 1. Генерируем большое простое число p нужной битовой длины.
	m_p = generatePrime(m_key_bits, rng);

	// 2. Выбираем генератор g.
	// Для простоты реализации часто выбирают небольшое число. 
	m_g = 2;
}
\end{nvimstyle}
\begin{nvimstyle}
void ElGamal::generateKeys()
{
	boost::random::mt19937 rng(std::chrono::high_resolution_clock::now().time_since_epoch().count());

	// 1. Выбираем случайный секретный ключ d 
	// в диапазоне 1 < d < p-1. Используем p-1, так как размер подгруппы
	// может быть p-1 (если p - безопасное простое).
	boost::random::uniform_int_distribution<BigInt> dist(2, m_p - 2);
	m_d = dist(rng);

	// 2. Вычисляем открытый ключ e: e = g^d mod p.
	m_e = boost::multiprecision::powm(m_g, m_d, m_p);
}
\end{nvimstyle}
\begin{nvimstyle}
#ifndef DIFFIE_HELLMAN_HPP
#define DIFFIE_HELLMAN_HPP

#include "protocol.hpp"

#include <boost/multiprecision/cpp_int.hpp>
#include <boost/random/mersenne_twister.hpp>

using BigInt = boost::multiprecision::cpp_int;

namespace CRYPTO
{
class DiffieHellman final : public Protocol
{
public:
	explicit DiffieHellman(unsigned int prime_bits = 512);
	void init() override;
\end{nvimstyle}

\begin{nvimstyle}
	QString encrypt(const QString& plaintext) override;
	QString decrypt(const QString& ciphertext) override;

	void generateSharedSecret();

	BigInt getPrime();
	BigInt getPublicKey() const;

	void setOtherPartyPublicKey(const BigInt& key);
	void setOtherPartyPrime(const BigInt& prime);

private:
	void generateParameters();
	void generateKeys();
    
	BigInt generatePrime(unsigned int bits, boost::random::mt19937& rng);

private:
	// --- Параметры и ключи Диффи-Хеллмана ---
	unsigned int m_prime_bits;

	// Публичные параметры, общие для обеих сторон
	BigInt m_p; // Простое число-модуль `p`
	BigInt m_g; // Первообразный корень (генератор) `g`

	// Ключи этого участника
	BigInt m_private_key; // Закрытый ключ (секретное число `a` или `b`)
	BigInt m_public_key;  // Открытый ключ (A = g^a mod p или B = g^b mod p)

	// Данные, полученные от другого участника
	BigInt m_other_party_public_key; // Открытый ключ другого участника (A или B)

	// Финальный результат
	BigInt m_shared_secret; // Общий секретный ключ `s = (B^a) mod p = (A^b) mod p`
};

} // namespace CRYPTO

#endif // DIFFIE_HELLMAN_HPP
\end{nvimstyle}
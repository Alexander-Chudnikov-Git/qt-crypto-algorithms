\begin{nvimstyle}
QString ElGamal::encrypt(const QString& plaintext)
{
	if (m_p == 0 || m_g == 0 || m_e == 0)
	{
		throw std::runtime_error("ElGamal is not initialized. Call init() before encryption.");
	}

	// 1. Преобразуем открытый текст в BigInt s (через hex, как в RSA).
	QByteArray bytes		 = plaintext.toUtf8();
	QString	   hex_plaintext = bytes.toHex();
	BigInt	   s("0x" + hex_plaintext.toStdString());

	if (s >= m_p)
	{
		throw std::runtime_error("Message is too large for the current key size.");
	}
\end{nvimstyle}
\begin{nvimstyle}
	// 2. Выбираем случайное сессионное (эфемерное) число k: 1 < k < p-1.
	boost::random::mt19937							rng(std::chrono::high_resolution_clock::now().time_since_epoch().count());
	boost::random::uniform_int_distribution<BigInt> dist(2, m_p - 2);
	BigInt											k = dist(rng);

	// 3. Вычисляем первую компоненту шифртекста: r = g^k mod p.
	BigInt r = boost::multiprecision::powm(m_g, k, m_p);

	// 4. Вычисляем вторую компоненту шифртекста: c = s * e^k mod p.
	BigInt e_k = boost::multiprecision::powm(m_e, k, m_p);
	BigInt c   = (s * e_k) % m_p;

	// 5. Формируем шифртекст: пара (r, c) в виде hex-строк, разделенных пробелом.
	QString r_hex = QString::fromStdString(r.str(0, std::ios_base::hex));
	QString c_hex = QString::fromStdString(c.str(0, std::ios_base::hex));

	return r_hex + " " + c_hex;
}
\end{nvimstyle}
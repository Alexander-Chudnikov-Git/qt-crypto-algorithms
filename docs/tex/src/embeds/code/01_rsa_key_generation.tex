
\begin{nvimstyle}
BigInt RSA::generatePrime(unsigned int bits, boost::random::mt19937& rng)
{
	// Задаем диапазон для генерации числа с нужным количеством бит
	BigInt											lower_bound = BigInt(1) << (bits - 1);
	BigInt											upper_bound = (BigInt(1) << bits) - 1;
	boost::random::uniform_int_distribution<BigInt> dist(lower_bound, upper_bound);

	BigInt candidate;
	while (true)
	{
		candidate = dist(rng);
		// Убедимся, что число нечетное
		if (candidate % 2 == 0)
		{
			candidate++;
		}
		// Проверяем на простоту с помощью теста Миллера-Рабина
		// 25 итераций дают очень высокую вероятность того, что число простое
		if (boost::multiprecision::miller_rabin_test(candidate, 25))
		{
			return candidate;
		}
	}
}
\end{nvimstyle}

\begin{nvimstyle}
void RSA::generateKeys()
{
    // 1. Инициализация генератора случайных чисел
    boost::random::mt19937 rng(std::chrono::high_resolution_clock::now().time_since_epoch().count());

    // 2. Генерация двух различных простых чисел p и q
    unsigned int prime_bits = m_key_bits / 2;
    do
    {
        m_p = generatePrime(prime_bits, rng);
        m_q = generatePrime(prime_bits, rng);
    } while (m_p == m_q);

    // 3. Вычисление модуля m и функции Эйлера phi(m)
    m_m   = m_p * m_q;
    m_phi = (m_p - 1) * (m_q - 1);

    // 4. Выбор открытой экспоненты e
    m_e = 65537;
    if (boost::multiprecision::gcd(m_e, m_phi) != 1)
    {
        // В маловероятном случае, если 65537 не подходит,
        // генерируем ключи заново
        generateKeys();
        return;
    }

    // 5. Вычисление закрытой экспоненты d
    m_d = inverse(m_e, m_phi);
}
\end{nvimstyle}
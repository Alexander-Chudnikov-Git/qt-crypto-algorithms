\begin{nvimstyle}
#ifndef RSA_HPP
#define RSA_HPP

#include "protocol.hpp"

#include <boost/multiprecision/cpp_int.hpp>
#include <boost/random/mersenne_twister.hpp>

using BigInt = boost::multiprecision::cpp_int;

namespace CRYPTO
{
class RSA final : public Protocol
{
public:
	explicit RSA(unsigned int key_bits = 2048);

	void init() override;

	QString encrypt(const QString& plaintext) override;
	QString decrypt(const QString& ciphertext) override;

private:
	void generateKeys();
	BigInt generatePrime(unsigned int bits, boost::random::mt19937& rng);

	// Параметры RSA
	unsigned int m_key_bits;
	BigInt		 m_m; // Модуль m = p * q (часть открытого ключа)
	BigInt		 m_e; // Открытая экспонента e (часть открытого ключа)
	BigInt		 m_d; // Закрытая экспонента d (закрытый ключ)

	// Простые множители сохраняются для потенциального использования, но являются частью закрытого ключа
	BigInt m_p, m_q;
	BigInt m_phi; // Функция Эйлера: phi(n) = (p-1)*(q-1)
};
} // namespace CRYPTO

#endif // RSA_HPP

\end{nvimstyle}
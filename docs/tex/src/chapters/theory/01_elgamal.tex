\subsection{Схема шифрования Эль-Гамаля}

Схема шифрования Эль-Гамаля, предложенная Тахиром Эль-Гамалем в 1984 году, является асимметричной криптосистемой, основанной на сложности решения задачи дискретного логарифмирования в конечных полях. В отличие от RSA, это вероятностный алгоритм: одно и то же открытое сообщение может быть зашифровано в множество различных шифртекстов, что повышает его стойкость к криптоанализу.\\

\noindent Рассмотрим реализацию схемы в мультипликативной группе конечного поля $\mathbb{F}_p^*$.

\subsubsection{Генерация публичных параметров}
Перед генерацией ключей для пользователей необходимо создать общий набор параметров, которые будут известны всем участникам системы:
\begin{enumerate}
    \item Выбирается большое простое число $p$.
    \item Выбирается целое число $g$ (называемое генератором или порождающим элементом), которое является порождающим элементом циклической подгруппы порядка $q$ в группе $\mathbb{F}_p^*$.
\end{enumerate}
Параметры $(p, q, g)$ являются открытыми и могут использоваться всеми участниками.

\subsubsection{Алгоритм генерации ключевой пары}
Каждый пользователь (получатель сообщения) генерирует свою пару ключей:
\begin{enumerate}
    \item Выбирается случайное целое число $d$ (секретный ключ), удовлетворяющее условию $1 < d < q$.
    \item Вычисляется открытый ключ $e$ по формуле:
    \[ e \equiv g^d \pmod p. \]
\end{enumerate}
Таким образом:
\begin{itemize}
    \item \textbf{Открытый ключ} — это кортеж $(e, p, q, g)$.
    \item \textbf{Закрытый ключ} — это число $d$.
\end{itemize}

\subsubsection{Алгоритм шифрования}
Для шифрования сообщения $s$ (представленного в виде числа, $1 < s < p$), отправитель использует открытый ключ получателя:
\begin{enumerate}
    \item Выбирается случайное сессионное (эфемерное) число $k$, такое что $1 < k < q$.
    \item Вычисляется первая компонента шифртекста:
    \[ r \equiv g^k \pmod p. \]
    \item Вычисляется вторая компонента шифртекста:
    \[ c \equiv s \cdot e^k \pmod p. \]
\end{enumerate}
Шифртекстом является пара чисел $(r, c)$.

\subsubsection{Алгоритм расшифрования}
Получив шифртекст $(r, c)$, получатель использует свой секретный ключ $d$ для восстановления исходного сообщения $s$:
\[ s \equiv c \cdot r^{-d} \pmod p. \]
Здесь $r^{-d}$ — это мультипликативное обратное к $r^d$ по модулю $p$.\\

\noindent Корректность расшифрования обеспечивается следующими преобразованиями:
\[ c \cdot r^{-d} \equiv (s \cdot e^k) \cdot (g^k)^{-d} \equiv s \cdot (g^d)^k \cdot g^{-kd} \equiv s \cdot g^{dk} \cdot g^{-dk} \equiv s \cdot g^{0} \equiv s \pmod p. \]

\subsubsection{Криптостойкость и особенности}
Безопасность схемы Эль-Гамаля целиком основывается на вычислительной сложности \textbf{задачи дискретного логарифмирования (DLP)}. Зная открытые параметры $(p, q, g)$ и открытый ключ $e$, злоумышленник для нахождения секретного ключа $d$ должен решить уравнение $e \equiv g^d \pmod p$. Для достаточно больших $p$ и $q$ эта задача считается неразрешимой за приемлемое время.\\

\noindent Схема имеет две важные особенности:
\begin{itemize}
    \item \textbf{Вероятностный характер.} Из-за использования случайного числа $k$ при каждом шифровании, одно и то же сообщение $s$ преобразуется в разные шифртексты. Это свойство защищает от атак, основанных на частотном анализе.
    \item \textbf{Удвоение размера.} Шифртекст $(r, c)$ состоит из двух чисел, поэтому его размер вдвое превышает размер исходного сообщения $s$. Это является основным недостатком схемы, ограничивающим её применение для шифрования больших объёмов данных.
\end{itemize}
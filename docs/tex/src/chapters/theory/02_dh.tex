\subsection{Протокол обмена ключами Диффи–Хеллмана}

Протокол Диффи–Хеллмана (Diffie–Hellman, DH), разработанный Уитфилдом Диффи и Мартином Хеллманом в 1976 году, стал первым в истории практическим методом для установления общего секретного ключа по незащищённому каналу связи. В отличие от RSA и Эль-Гамаля, DH не является схемой шифрования, а предназначен исключительно для выработки общего секрета, который затем может быть использован в симметричных алгоритмах шифрования.

\subsubsection{Генерация публичных параметров}
Как и в схеме Эль-Гамаля, участники должны предварительно согласовать общие параметры:
\begin{enumerate}
    \item Большое простое число $p$.
    \item Порождающий элемент (генератор) $g$ циклической группы по модулю $p$.
\end{enumerate}
Эти параметры $(p, g)$ являются открытыми и не требуют защиты.

\subsubsection{Алгоритм выработки общего секрета}
Процесс выработки ключа между двумя участниками, А и Б, происходит следующим образом:
\begin{enumerate}
    \item \textbf{Сторона А} выбирает случайное секретное число $x$ (где $1 < x < p-1$), вычисляет $K_A = g^x \pmod p$ и отправляет результат $K_A$ стороне Б.
    
    \item \textbf{Сторона Б} выбирает своё случайное секретное число $y$ (где $1 < y < p-1$), вычисляет $K_B = g^y \pmod p$ и отправляет результат $K_B$ стороне А.
    
    \item \textbf{Сторона А}, получив $K_B$, вычисляет общий секретный ключ $k$ по формуле:
    \[ k \equiv (K_B)^x \equiv (g^y)^x \equiv g^{yx} \pmod p. \]
    
    \item \textbf{Сторона Б}, получив $K_A$, вычисляет тот же самый ключ:
    \[ k \equiv (K_A)^y \equiv (g^x)^y \equiv g^{xy} \pmod p. \]
\end{enumerate}
В результате оба участника получают одинаковый секретный ключ $k$, не передавая в открытом виде свои секретные числа $x$ и $y$.

\subsubsection{Криптостойкость}
Безопасность протокола Диффи–Хеллмана основана на сложности решения \textbf{задачи дискретного логарифмирования (DLP)}. Злоумышленник, перехватывающий сообщения в канале, имеет доступ к числам $p, g, K_A = g^x$ и $K_B = g^y$. Чтобы вычислить общий секрет $k = g^{xy}$, ему необходимо найти либо секретное число $x$ из $K_A$, либо $y$ из $K_B$. Эта задача, как и в случае с Эль-Гамалем, считается вычислительно неразрешимой для достаточно больших $p$.

\subsubsection{Уязвимость: атака «человек посередине» (Man-in-the-Middle)}
Базовый протокол Диффи–Хеллмана уязвим к атаке «человек посередине» (Man-in-the-Middle, MitM), поскольку он не аутентифицирует участников обмена. Злоумышленник C может вклиниться в канал связи между A и Б.

Атака проходит следующим образом:
\begin{enumerate}
    \item Участник А отправляет $g^x \pmod p$ в сторону Б. Злоумышленник С перехватывает это сообщение.
    \item Злоумышленник С генерирует собственное секретное число $z_1$, вычисляет $g^{z_1} \pmod p$ и отправляет его участнику Б, выдавая себя за А.
    \item Участник Б отправляет $g^y \pmod p$ в сторону А. С также перехватывает это сообщение.
    \item Злоумышленник С генерирует второе секретное число $z_2$, вычисляет $g^{z_2} \pmod p$ и отправляет его участнику А, выдавая себя за Б.
\end{enumerate}
В результате складывается следующая ситуация:
\begin{itemize}
    \item Участник А устанавливает общий ключ со злоумышленником С: $k_{AC} = (g^{z_2})^x = g^{xz_2} \pmod p$.
    \item Участник Б также устанавливает общий ключ со злоумышленником С: $k_{BC} = (g^{z_1})^y = g^{yz_1} \pmod p$.
\end{itemize}
При этом А и Б не догадываются, что выработали ключи не друг с другом, а с нарушителем. Злоумышленник С, зная оба ключа, может перехватывать, читать и изменять все сообщения, которыми обмениваются А и Б, сохраняя при этом иллюзию защищённого канала. Для защиты от этой атаки требуются дополнительные механизмы аутентификации, например, цифровые подписи.
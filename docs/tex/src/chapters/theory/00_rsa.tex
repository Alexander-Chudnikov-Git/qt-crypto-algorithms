\subsection{Схема шифрования RSA}

\textbf{RSA} (Rivest–Shamir–Adleman) — это асимметричная криптографическая система, основанная на вычислительной сложности задачи факторизации больших целых чисел. Предложенная в 1977 году Роном Ривестом, Ади Шамиром и Леонардом Адлеманом, она стала одной из первых практически применимых систем с открытым ключом и на сегодняшний день является мировым стандартом. Интересно, что аналогичный алгоритм был разработан ранее, в 1973 году, британским инженером Клиффордом Коксом (Clifford Cocks), но его работа была засекречена.\\

\noindent Принцип работы RSA заключается в использовании двух ключей: открытого, который можно свободно распространять, и закрытого (секретного), который должен храниться в тайне.

\subsubsection{Алгоритм генерации ключевой пары}
Для создания открытого и закрытого ключей выполняются следующие шаги:

\begin{enumerate}
    \item \textbf{Выбор простых чисел.} Выбираются два различных очень больших простых числа $p$ и $q$.
    
    \item \textbf{Вычисление модуля.} Вычисляется их произведение, которое называется модулем: 
    \[ m = p \cdot q. \]
    
    \item \textbf{Вычисление функции Эйлера.} Вычисляется значение функции Эйлера от числа $m$:
    \[ \varphi(m) = (p-1)(q-1). \]
    Именно это значение, а не сам модуль $m$, определяет криптографическую стойкость ключа.
    
    \item \textbf{Выбор открытой экспоненты.} Выбирается целое число $e$ (открытая экспонента), удовлетворяющее двум условиям:
    \[ 1 < e < \varphi(m) \quad \text{и} \quad \text{НОД}(e, \varphi(m)) = 1. \]
    Часто в качестве $e$ выбирают небольшие простые числа, например 65537 ($2^{16}+1$), для ускорения процесса шифрования.
    
    \item \textbf{Вычисление секретной экспоненты.} Вычисляется число $d$ (секретная экспонента), которое является мультипликативно обратным к $e$ по модулю $\varphi(m)$:
    \[ ed \equiv 1 \pmod{\varphi(m)}. \]
    Число $d$ находится с помощью расширенного алгоритма Евклида.
    
    \item \textbf{Формирование ключей.}
    \begin{itemize}
        \item \textbf{Открытый ключ} — это пара чисел $(m, e)$.
        \item \textbf{Закрытый ключ} — это пара чисел $(m, d)$.
    \end{itemize}
    Простые числа $p$ и $q$ после генерации ключей должны быть уничтожены.
\end{enumerate}

\subsubsection{Алгоритм шифрования}
Чтобы зашифровать сообщение $s$, представленное в виде целого числа ($1 < s < m$), отправитель (абонент А) должен выполнить следующую операцию, используя открытый ключ получателя (абонента Б):
\[ c \equiv s^e \pmod m. \]
Полученное число $c$ является шифртекстом и передаётся по открытому каналу связи.

\subsubsection{Алгоритм расшифрования}
Для расшифрования шифртекста $c$ получатель (абонент Б) использует свой секретный ключ $(m, d)$:
\[ s \equiv c^d \pmod m. \]
Корректность расшифрования следует из теоремы Эйлера. Поскольку $ed \equiv 1 \pmod{\varphi(m)}$, то $ed = k\varphi(m) + 1$ для некоторого целого $k$. Тогда:
\[ c^d \equiv (s^e)^d = s^{ed} = s^{k\varphi(m)+1} \equiv (s^{\varphi(m)})^k \cdot s^1 \equiv 1^k \cdot s \equiv s \pmod m. \]
Таким образом, получатель восстанавливает исходное сообщение $s$.

\subsubsection{Криптографическая стойкость}
Стойкость криптосистемы RSA основывается на вычислительной сложности задачи факторизации больших целых чисел. Зная только открытый ключ $(m, e)$, злоумышленник для нахождения секретного ключа $d$ должен сначала вычислить значение $\varphi(m)=(p-1)(q-1)$. Для этого ему необходимо разложить модуль $m$ на простые сомножители $p$ и $q$.\\

\noindent Более того, можно доказать, что знание секретного ключа $d$ эквивалентно возможности факторизовать модуль $m$. Это основано на вероятностном алгоритме, который использует тот факт, что $ed - 1$ является кратным $\varphi(m)$. Представим $ed - 1$ в виде $2^n \cdot t$, где $t$ — нечётное число. Для случайным образом выбранного числа $a$ ($1 < a < m$) рассматривается последовательность $a^t, a^{2t}, \dots, a^{2^n t} \pmod m$. Последний элемент этой последовательности равен 1.\\

\noindent Если в этой последовательности удаётся найти элемент $v$, такой что $v \not\equiv \pm 1 \pmod m$, но $v^2 \equiv 1 \pmod m$, то найден нетривиальный квадратный корень из единицы. В этом случае делители $p$ и $q$ могут быть найдены как $\text{НОД}(v-1, m)$ и $\text{НОД}(v+1, m)$. Доказывается, что для случайно выбранного $a$ вероятность успеха такой атаки составляет не менее $\frac{1}{2}$. Несколько попыток с разными значениями $a$ почти гарантированно приводят к факторизации модуля.\\

\noindent Таким образом, безопасность RSA неразрывно связана со сложностью задачи факторизации, и раскрытие секретного ключа $d$ полностью компрометирует систему.
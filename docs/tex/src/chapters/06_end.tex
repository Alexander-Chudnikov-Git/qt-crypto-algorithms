\section{Заключение}

\noindent
В рамках данной курсовой работы была успешно выполнена задача по исследованию, практической реализации и сравнительному анализу трех фундаментальных криптографических примитивов: асимметричных шифросистем RSA и Эль-Гамаля, а также протокола обмена ключами Диффи-Хеллмана.

Работа была структурирована в три логических этапа:
\begin{itemize}
    \item \textbf{Теоретическое исследование}, в ходе которого были рассмотрены математические основы, алгоритмы генерации ключей, шифрования, расшифрования, а также криптографическая стойкость и уязвимости каждой системы.
    \item \textbf{Программная реализация} на языке C++ с использованием объектно-ориентированного подхода. Создание общего базового класса \texttt{Protocol} позволило обеспечить унифицированный интерфейс и продемонстрировать модульность и расширяемость системы. Для работы с большими числами была использована библиотека Boost.Multiprecision, а для создания интерактивного демонстрационного приложения~— фреймворк Qt.
    \item \textbf{Тестирование и анализ}, ставшие кульминацией работы. Было проведено комплексное тестирование производительности реализованных алгоритмов для размеров ключей в диапазоне от 8 до 4096~бит.
\end{itemize}

По результатам тестирования были сделаны следующие ключевые выводы:
\begin{itemize}
    \item \textbf{Генерация ключей:} RSA демонстрирует наилучшую производительность и предсказуемый рост времени. Эль-Гамаль является значительно более затратным на этом этапе, в то время как полный цикл обмена ключами Диффи-Хеллмана, будучи быстрым на малых ключах, показывает наихудшую масштабируемость.
    \item \textbf{Шифрование:} RSA обладает подавляющим преимуществом в скорости шифрования благодаря использованию малой публичной экспоненты ($e=65537$). Шифрование в схеме Эль-Гамаля является вычислительно сложной операцией, что делает его на несколько порядков медленнее.
    \item \textbf{Расшифрование:} Производительность операций расшифрования для RSA и Эль-Гамаля оказалась практически идентичной. Это подтверждает теоретическое положение о том, что основная вычислительная нагрузка в обеих системах ложится на владельца секретного ключа, так как требует модульного возведения в степень с использованием большой секретной экспоненты.
\end{itemize}

\vspace{1cm}

\noindent
Программная реализация протоколов на языке С++, использованная в данной работе, располагается в открытом доступе на GitHub по ссылке:

\vspace{1cm}
\begin{center}
    \textit{[https://github.com/Alexander-Chudnikov-Git/qt-crypto-algorithms.git]}
\end{center}
\vspace{1cm}
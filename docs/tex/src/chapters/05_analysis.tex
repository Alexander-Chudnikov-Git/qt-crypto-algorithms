\section{Тестирование и анализ}

В данном разделе представлен анализ производительности криптографических алгоритмов RSA, ElGamal и протокола обмена ключами Diffie-Hellman. Тестирование проводилось для размеров ключей от 8 до 4096 бит с шагом, равным удвоению предыдущего значения. \\

\noindent Для наглядности все графики построены в логарифмическом масштабе по обеим осям (log-log). Ось X (размер ключа) имеет основание логарифма 2, что позволяет равномерно распределить тестовые точки. Ось Y (время выполнения) использует натуральный логарифм для эффективного отображения значений в широком диапазоне — от долей миллисекунды до десятков секунд.

\subsection{Генерация и обмен ключами}
На первом этапе сравнивается время, необходимое для создания ключевого материала. Для RSA и ElGamal это процесс генерации пары ключей, а для Diffie-Hellman — полный цикл обмена, в результате которого обе стороны получают общий секрет.

\begin{figure}[h!]
\centering
\begin{tikzpicture}
\begin{axis}[
    title={Производительность генерации и обмена ключами},
    xlabel={Размер ключа, бит},
    ylabel={Время выполнения, мс},
    xmode=log,
    log basis x=2,
    ymode=log,
    legend pos=north west,
    grid=major,
    width=0.9\textwidth,
    height=8cm,
    xtick={8, 16, 32, 64, 128, 256, 512, 1024, 2048, 4096},
    xticklabels={8, 16, 32, 64, 128, 256, 512, 1024, 2048, 4096},
]

% Данные для RSA
\addplot+[smooth, mark=*] coordinates {
    (256, 5) (512, 35) (1024, 174) (2048, 910) (4096, 14738)
};
\addlegendentry{RSA - Генерация}

% Данные для ElGamal
\addplot+[smooth, mark=square*] coordinates {
    (128, 3) (256, 9) (512, 84) (1024, 546) (2048, 7610) (4096, 42693)
};
\addlegendentry{ElGamal - Генерация}

% Данные для Diffie-Hellman
\addplot+[smooth, mark=triangle*] coordinates {
    (8, 30) (16, 46) (32, 44) (64, 31) (128, 38) (256, 31) (512, 72) (1024, 949) (2048, 13316) (4096, 63257)
};
\addlegendentry{DH - Обмен ключами}

\end{axis}
\end{tikzpicture}
\caption{Сравнение времени генерации ключей и полного цикла обмена DH.}
\label{fig:key_generation}
\end{figure}

\paragraph{Анализ графика:}
График наглядно демонстрирует экспоненциальную зависимость времени генерации от размера ключа для всех алгоритмов.
\begin{itemize}
    \item \textbf{Diffie-Hellman} показывает интересное поведение: для малых ключей (до 512 бит) он является одним из самых быстрых, но его производительность резко деградирует при увеличении ключа, делая его самым медленным на размерах 2048 и 4096 бит.
    \item \textbf{RSA} демонстрирует стабильный и предсказуемый рост. Он немного быстрее ElGamal.
    \item \textbf{ElGamal} является самым затратным по времени на этапе генерации ключей среди асимметричных шифров, что особенно заметно на больших размерах ключа.
\end{itemize}

\clearpage

\subsection{Шифрование}
На этом графике сравнивается производительность операций шифрования для RSA и ElGamal. Diffie-Hellman здесь не представлен, так как он является протоколом обмена ключами, а не шифрования.

\begin{figure}[h!]
\centering
\begin{tikzpicture}
\begin{axis}[
    title={Производительность шифрования},
    xlabel={Размер ключа, бит},
    ylabel={Время выполнения, мс},
    xmode=log,
    log basis x=2,
    ymode=log,
    legend pos=north west,
    grid=major,
    width=0.9\textwidth,
    height=8cm,
    xtick={128, 256, 512, 1024, 2048, 4096},
]

% Данные для RSA
% Нулевые значения не отображаются на логарифмической шкале
\addplot+[smooth, mark=*] coordinates {
    % Пропускаем (128,0), (256,0), (512,0), (1024,0), (2048,0), (4096,0)
    % Можно добавить фиктивную точку для наглядности
    (128, 0.1) (4096, 0.1)
};
\addlegendentry{RSA - Шифрование}

% Данные для ElGamal
\addplot+[smooth, mark=square*] coordinates {
    (512, 2) (1024, 15) (2048, 97) (4096, 621)
};
\addlegendentry{ElGamal - Шифрование}

\end{axis}
\end{tikzpicture}
\caption{Сравнение времени шифрования.}
\label{fig:encryption}
\end{figure}

\paragraph{Анализ графика:}
Здесь наблюдается колоссальная разница в производительности.
\begin{itemize}
    \item \textbf{RSA} показывает практически мгновенное шифрование (в логах 0 мс). Это объясняется использованием фиксированной, небольшой публичной экспоненты ($e=65537$), что делает операцию возведения в степень чрезвычайно быстрой. На графике это представлено как плоская линия на минимальном уровне.
    \item \textbf{ElGamal}, напротив, требует выполнения двух сложных модульных возведений в степень для каждой операции шифрования. Это делает его значительно медленнее RSA, и его производительность также экспоненциально зависит от размера ключа.
\end{itemize}

\subsection{Расшифрование}
Последний график сравнивает время, необходимое для расшифрования сообщения.

\begin{figure}[h!]
\centering
\begin{tikzpicture}
\begin{axis}[
    title={Производительность расшифрования},
    xlabel={Размер ключа, бит},
    ylabel={Время выполнения, мс},
    xmode=log,
    log basis x=2,
    ymode=log,
    legend pos=north west,
    grid=major,
    width=0.9\textwidth,
    height=8cm,
    xtick={128, 256, 512, 1024, 2048, 4096},
]

% Данные для RSA
\addplot+[smooth, mark=*] coordinates {
    (512, 1) (1024, 7) (2048, 49) (4096, 304)
};
\addlegendentry{RSA - Расшифрование}

% Данные для ElGamal
\addplot+[smooth, mark=square*] coordinates {
    (512, 1) (1024, 8) (2048, 49) (4096, 308)
};
\addlegendentry{ElGamal - Расшифрование}

\end{axis}
\end{tikzpicture}
\caption{Сравнение времени расшифрования.}
\label{fig:decryption}
\end{figure}

\paragraph{Анализ графика:}
В отличие от шифрования, производительность расшифрования для RSA и ElGamal оказывается \textbf{практически идентичной}.
\begin{itemize}
    \item Это связано с тем, что в обоих случаях операция расшифрования является наиболее вычислительно сложной и требует модульного возведения в степень с использованием большой секретной экспоненты (секретного ключа $d$ в RSA и $x$ в ElGamal). 
    \item Небольшие расхождения в значениях можно отнести на счет погрешностей измерения и особенностей конкретных сгенерированных ключей.
    \item Это подтверждает теоретические выкладки о том, что основная вычислительная нагрузка в асимметричных системах, как правило, ложится на владельца секретного ключа.
\end{itemize}

